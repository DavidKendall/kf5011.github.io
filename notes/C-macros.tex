%&pdflatex
\documentclass[xcolor=svgnames]{beamer}
\usepackage[british]{babel}

\usepackage{minted}
\usepackage{sourcecodepro}
\usepackage[T1]{fontenc}

\usepackage{tcolorbox}
\usepackage{graphicx}
\usepackage{booktabs}

\usetheme[block=fill,progressbar=frametitle]{metropolis}
\usepackage{lmodern}

\usepackage{bytefield}

\title{Control systems and Computer Networks}
\subtitle{The C pre-processor \& C Macros}

\author{Dr Alun Moon}
\date{Lecture 1.c}
\begin{document}
\frame{\maketitle}

\begin{frame}{The C Pre-Processor}
The C pre-processor has been a part of the C standard since the beginning of C.

It acts to transform the source code according to it's rules, prior to the modified code being fed to the compiler.

\begin{alertblock}{Note Well}
    This is a \alert{modification} of the source code.  The Compiler will report errors in the code \alert{after} the pre-processor has modified the code
\end{alertblock}

Errors introduced via the pre-processor can be \alert{\emph{\textbf{very} hard}} to debug!
\end{frame}

\begin{frame}{Syntax}
The syntax is very simple,
\begin{enumerate}
    \item lines beginning with a hash character \alert{\#} are pre-processor directives
    \item tokens are split using white-space
\end{enumerate}

\end{frame}

\begin{frame}[fragile]{Includes}
\begin{tcolorbox}
\begin{minted}{c}
#include <stdio.h>
#include "library.h"
\end{minted}
\end{tcolorbox}

The \mintinline{c}{#include} directive \alert{\emph{copies in}} the file given, at that point in the source code.

By convention these files are given the \texttt{.h} suffix, and are called \alert{\emph{header files}}

Files are searched for using two rules given by the kind of quotes used
\begin{itemize}
    \item[\texttt{<>}] looks for standard header files in a system defined place (the include-path)
    \item[\texttt{""}] looks for files in the same directory as the original source code
\end{itemize}

\end{frame}

\begin{frame}[fragile]{Define}{Simple version}
\begin{tcolorbox}
\begin{minted}[frame=leftline]{c}
#define BUFFSIZE 1024
#define pi 3.141592654
\end{minted}
\end{tcolorbox}

The simple version of \mintinline{c}{#define} creates a token and a \alert{\emph{textual}} substitution

Where the pre-processor finds a matching token in the source code, it is replaced with the text.

\begin{tabular}{l@{$\rightarrow$}l}
\mintinline{c}{char inputbuffer[BUFFSIZE];}%
&
\mintinline{c}{char inputbuffer[1024];}
\\
\mintinline{c}{A = pi*r*r} & \mintinline{c}{A = 3.141592654*r*r}\\
\end{tabular}

\end{frame}

\end{document}
%%%%======================
