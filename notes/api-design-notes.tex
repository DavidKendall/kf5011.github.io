%&pdflatex
\documentclass{tufte-handout}
\usepackage[british]{babel}

\usepackage{tikz}
\usetikzlibrary{positioning}

\usepackage{minted}
\usepackage{booktabs}
\usepackage{tcolorbox}

\title{API Design Notes}
\author{Dr Alun Moon}
\begin{document}
\maketitle
\begin{abstract}
This describes the design of the api for the
FRDM k64F
\end{abstract}
\section{Introduction}
The layer model for the API places Port and Bit access at the
bottom.

\section{Memory model and C}
The Bit-Band alias region allows direct access to bits, each word is mapped to a bit in the peripheral region.

\begin{table}
    \caption{Peripheral Memory }
    \label{}
    \begin{tabular}{lrr}
        & start address & end address \\
Peripheral memory region  & \texttt{40000000} & \texttt{400FFFFF} \\
Bit band   & \texttt{42000000} & \texttt{43FFFFFF} \\
    \end{tabular}
\end{table}
\begin{minted}{c}
typedef uint32_t portregisters[16];/* works but clunky */
typedef uint32_t registerbits[32];
typedef registerbits portbits[16];
portregisters *ports = (portregisters*)(0x400FF000);
portbits *bits = (portbits*)(0x49FE0000);

ports[PortB][Clear];
bits[PortB][Clear][22];
\end{minted}

\section{Port}
The MCU has 5 ports: A, B, C, D, and E.  Their default state is inactive.
To enable a Port the5 clock signal has to be distributed to the Port,
via the System Clock Gating Control Register 5 (SIM\_SCGC5).

\begin{table}
    \caption{System Clock Gating Control Register 5 (SIM\_SCGC5)}
    \label{simscgc5}
    \begin{tabular}{lrr}
SIM\_SCGC5 address  && \texttt{40048038} \\  \midrule
Port A  & 9 &  \texttt{42900724} \\
Port B  & 10 & \texttt{42900728} \\
Port C  & 11 & \texttt{4290072C} \\
Port D  & 12 & \texttt{42900730} \\
Port E  & 13 & \texttt{42900734} \\
    \end{tabular}
\end{table}

\section{Bits}
The port control registers control the behaviour of each bit.
The \texttt{MUX} field (bits 8--10) controls the function, for GPIO pins the MUX value is 1 (3 bits 001)

\begin{table}
    \caption{Device Bit addresses}
    \label{}
    \begin{tabular}{llrrrr}
        Device & port & bit & PCR & Output & Set \\ \midrule
Red LED & B & 22 & \texttt{4004A058} &
                           \texttt{43FE0858} & \texttt{42FE08D8} \\
Blue LED & B & 21 & \texttt{4004A054} &
                           \texttt{43FE0584} & \texttt{42FE08D4}
    \end{tabular}
\end{table}

\section{Sequence of events}
\begin{enumerate}
    \item open LED \hfill maps from LED to bit, including logic levels
    \item opens Bit for writing \hfill collates bit alias addresses
    \item opens Port  \hfill ensures clock signal is gated to enable port
\end{enumerate}

\subsection{Some semantics}
I want to have a common API,
\begin{tcolorbox}
\begin{minted}{c}
    result = open(MajorNo, MinorNo, params, params);
\end{minted}
\end{tcolorbox}
Where the Major device number is used to select the driver from a lookup table.
The device gets opened as.
\begin{tcolorbox}
\begin{minted}{c}
    result = dev_open(MinorNo, params, params);
\end{minted}
\end{tcolorbox}
The Minor number is used to identify the particular device.

\paragraph{LED} the parameters have no effect here, an LED can only be opened for writing.
\paragraph{BIT} we need to know the port and bit numbers, bits can be opened for reading or writing.   They can be GPIO or special purpose.
\paragraph{Port} As a minimum the port needs to have the clock signal gated to enable the port.  Do reading and writing have meaning?
\end{document}
