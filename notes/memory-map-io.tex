%&xelatex
\documentclass[xcolor=svgnames]{beamer}
\usepackage[british]{babel}
\usepackage[T1]{fontenc}

\usepackage{minted}

\usepackage{tcolorbox}
\usepackage{graphicx}
\usepackage{booktabs}

\usetheme[block=fill,progressbar=frametitle]{metropolis}
\usepackage{lmodern}

\usepackage{bytefield}

\useinnertheme{default}
\useoutertheme{infolines}
\usecolortheme{seahorse}
\setbeamercolor*{alerted text}{fg=blue!75!black}
\setbeamertemplate{blocks}[rounded]
\setbeamertemplate{itemize item}[triangle]
\setbeamertemplate{itemize subitem}[circle]
\setbeamertemplate{itemize subsubitem}[square]

\usecolortheme{rose}
\definecolor{NUblue}{RGB}{62,141,165}
\definecolor{NUbluedark}{RGB}{40,119,143}
\setbeamercolor*{palette primary}{use=structure,fg=white,bg=NUblue}
\setbeamercolor*{palette quaternary}{fg=white,bg=NUbluedark}
\setbeamercolor{section in head/foot}{fg=white,bg=NUbluedark}
\setbeamercolor{subsection in head/foot}{fg=white,bg=NUblue}
\setbeamercolor{frametitle}{fg=NUbluedark!150,bg=NUblue!40}
\setbeamercolor{title in head/foot}{fg=white,bg=NUblue}
\setbeamercolor{author in head/foot}{fg=white, bg=NUbluedark}
\setbeamercolor{date in head/foot}{fg=white, bg=NUblue!60}
\setbeamercolor{title}{fg=NUbluedark!150,bg=NUblue!30}
\setbeamercolor{date}{fg=NUbluedark!150}
\setbeamercolor{block title}{fg=white,bg=NUblue}


\title{Control systems and Computer Networks}
\subtitle{LEDs and Switches}

\author{Dr Alun Moon}
\date{Lecture 1}
\begin{document}
\frame{\maketitle}


\begin{frame}{Memory mapped IO}
\begin{itemize}
    \item Access to hardware is via read/writes to addresses
    \item Easier to build
    \item easier instruction set

\end{itemize}
\end{frame}

\begin{frame}{ARM}{Bit alias region}
    \begin{itemize}
        \item IO is via read/write to 32bit registers
        \item alias region
        \begin{itemize}
            \item read and write to each 32bit word
            \item reads and writes to each bit in the IO registers
        \end{itemize}
    \end{itemize}

\end{frame}

\begin{frame}[fragile]{Complex declarations}
    The complete declaration for a memory mapped I/O register is something like.
    \begin{tcolorbox}
        \begin{minted}{c}
volatile uint32_t *const IOmap;
        \end{minted}
    \end{tcolorbox}
    Which reads as\ldots
    \mintinline{c}{IOmap} is a \alert{Constant} \alert{Pointer} to an \alert{unsigned 32 bit integer} which is \alert{Volatile}
    \begin{description}
        \item[Constant pointer] the value of the pointer (address) is constant
        \item[Volatile] tells the compiler that the value at the address may change, so always fetch the value from memory.
    \end{description}
\end{frame}

\begin{frame}[fragile]{C arrays and pointers}
Arrays and pointers in C have a close relationship;
\begin{exampleblock}{Arrays}
\begin{minted}{c}
    int modes[12];  /* array of 12 integers */
    modes[5];       /* 5th element (count from 0) */
\end{minted}
\end{exampleblock}
\begin{exampleblock}{Pointers}
\begin{minted}{c}
    int *data; /* pointer to an integer */
    *data = 5; /* write to address */
    data+1;    /* pointer to the next integer */
\end{minted}
\end{exampleblock}
\begin{exampleblock}{Arrays and Pointers}
\begin{minted}{c}
    data = modes;  /* array name is a pointer */
    data[6] = modes[5]; /* pointers as arrays */
\end{minted}
\end{exampleblock}
\end{frame}

\begin{frame}[fragile]{Memory}
    We can model the memory as an array of bytes
    \begin{tcolorbox}
        \begin{minted}{c}
   uint8_t memory[SIZE];
        \end{minted}
    \end{tcolorbox}
    The ARM is a 32bit architecture and so it may be more
    convenient to model the memory as an array of 32bit words
    \begin{tcolorbox}
        \begin{minted}{c}
   uint32_t wordmemory[SIZE/4];
        \end{minted}
    \end{tcolorbox}

\end{frame}

\begin{frame}[fragile]{Pointer Arithmetic}
Given
\begin{minted}{c}
    uint32_t *word;
\end{minted}
Then\\
\begin{tabular}{ll}
    \mintinline{c}{word} & is an address aligned to 4 bytes \\
    \mintinline{c}{*word} & is an unsigned 32 bit integer at that address \\
    \mintinline{c}{word+1} & is the address of the \emph{next} integer, \alert{4 bytes on} \\
    \mintinline{c}{*(word+1)} & is the \alert{next} 32 bit integer\\
    \mintinline{c}{word[0]} & is the integer \alert{at} the address in \mintinline{c}{word}\\
    \mintinline{c}{word[1]} & is the next integer (at \mintinline{c}{word+1})
\end{tabular}
Pointer arithmetic (and arrays) take into account the \emph{size} of the thing pointed to.

\begin{exampleblock}{See also}
    the \mintinline{c}{sizeof} compile time operator
\end{exampleblock}
\end{frame}


\part{GPIO}
\frame\partpage

\begin{frame}{Port IO}
    Each port has

\begin{description}
    \item[Data out] sets the output
    \item[Set   ] writing 1 sets the output (sets to 1)
    \item[Clear ] writing 1 clears the output (sets to 0)
    \item[Toggle] writing 1 changes the output
    \item[Input ] reads the input
    \item[Direction] set the pin as output or input
\end{description}

\end{frame}

\begin{frame}{Port Addresses}
\begin{tabular}{rlrll}
    Port & Base address & register & offset & action\\\toprule
    Port A & 0x400FF000 & Data out & 0x00 & sets bits to 0 or 1\\
                    &&    Set      & 0x04 & 1 set bit,\\
                    &&&& 0 leaves bit unchanged \\
                    &&    Clear    & 0x08 & 1 clears bit \\
                    &&&& 0 leaves bit unchanged \\
                    &&    Toggle   & 0x0C & 1 toggles bit \\
                    &&&& 0 leaves bit unchanged \\
                    &&    Input    & 0x10 & reads bit state\\
                    &&   Direction & 0x14 & 1 is output, 0 is input \\\midrule
    Port B & 0x400FF040\\
    Port C & 0x400FF080\\
    Port D & 0x400FF0C0\\
    Port E & 0x400FF100\\
    \end{tabular}
\end{frame}

\begin{frame}{Endianness}
    Arrangement of bytes in a multi-byte value (4 bytes in a 32 bit integer)
\begin{description}
    \item[Big Endian] Most significant bytes come first in memory
    \item[Little Endian] Least significant bytes come first in memory
\end{description}
The ARM is \alert{Little} Endian
\begin{exampleblock}{For register/32bit integer at address \alert{\texttt{400FF004}}}
\begin{tabular}{lrr@{~--~}r}
Address & byte & \multicolumn{2}{c}{bits}\\
0x400FF004 &  0 &  0 & 7 \\
0x400FF005 &  1 &  8 & 15 \\
0x400FF006 &  2 &  16 & 23 \\
0x400FF007 &  3 &  24 & 31 \\
\end{tabular}
\end{exampleblock}
\end{frame}


\begin{frame}[fragile]{An initial model}
    We can have an initial model, thinking of the I/O memory as an array of words.

    \begin{tcolorbox}
        \begin{minted}{c}
enum registers {
    Output,Set,Clear,Toggle,Input,Direction
};
uint32_t *PortB = (uint32_t*)0x400FF00;
        \end{minted}
    \end{tcolorbox}

    The ports registers are now simply array access

    \begin{tcolorbox}
        \begin{minted}{c}
    PortB[Direction] |= 1<<22;
        \end{minted}
    \end{tcolorbox}
    Sets bit 22 in Port-B's direction register.
\end{frame}



\end{document}

%%%%======================
