%&pdflatex
\documentclass[xcolor=svgnames]{beamer}
\usepackage[british]{babel}

\usepackage{minted}
\usepackage{sourcecodepro}
\usepackage[T1]{fontenc}

\usepackage{tcolorbox}
\usepackage{graphicx}
\usepackage{booktabs}

\usetheme[block=fill,progressbar=frametitle]{metropolis}
\usepackage{lmodern}

\usepackage{bytefield}

\title{Control systems and Computer Networks}
\subtitle{LEDs and Switches}

\author{Dr Alun Moon}
\date{Lecture 1}
\begin{document}
\frame{\maketitle}


\begin{frame}{Memory mapped IO}
\begin{itemize}
    \item Access to hardware is via read/writes to addresses
    \item Easier to build
    \item easier instruction set

\end{itemize}
\end{frame}

\begin{frame}{ARM}{Bit alias region}
    \begin{itemize}
        \item IO is via read/write to 32bit registers
        \item alias region
        \begin{itemize}
            \item read and write to each 32bit word
            \item reads and writes to each bit in the IO registers
        \end{itemize}
    \end{itemize}

\end{frame}

\begin{frame}{Port IO}
    Each port has

\begin{description}
    \item[Data out] sets the output
    \item[Set   ] writing 1 sets the output (sets to 1)
    \item[Clear ] writing 1 clears the output (sets to 0)
    \item[Toggle] writing 1 changes the output
    \item[Input ] reads the input
    \item[Direction] set the pin as output or input
\end{description}

\end{frame}

\begin{frame}{Port Addresses}
\begin{tabular}{rlrll}
    Port & Base address & register & offset & action\\\toprule
    Port A & 0x400FF000 & Data out & 0x00 & sets bits to 0 or 1\\
                    &&    Set      & 0x04 & 1 set bit,\\
                    &&&& 0 leaves bit unchanged \\
                    &&    Clear    & 0x08 & 1 clears bit \\
                    &&&& 0 leaves bit unchanged \\
                    &&    Toggle   & 0x0C & 1 toggles bit \\
                    &&&& 0 leaves bit unchanged \\
                    &&    Input    & 0x10 & reads bit state\\
                    &&   Direction & 0x14 & 1 is output, 0 is input \\\midrule
    Port B & 0x400FF040\\
    Port C & 0x400FF080\\
    Port D & 0x400FF0C0\\
    Port E & 0x400FF100\\
    \end{tabular}
\end{frame}

\begin{frame}{Endianness}

\end{frame}

\begin{frame}[fragile]{C arrays and pointers}
Arrays and pointers in C have a close relationship;
\begin{exampleblock}{Arrays}
\begin{minted}{c}
    int modes[12];  /* array of 12 integers */
    modes[5];       /* 5th element (count from 0) */
\end{minted}
\end{exampleblock}
\begin{exampleblock}{Pointers}
\begin{minted}{c}
    int *data; /* pointer to an integer */
    *data = 5; /* write to address */
    data+1;    /* pointer to the next integer */
\end{minted}
\end{exampleblock}
\begin{exampleblock}{Arrays and Pointers}
\begin{minted}{c}
    data = modes;  /* array name is a pointer */
    data[6] = modes[5]; /* pointers as arrays */
\end{minted}
\end{exampleblock}
\end{frame}

\begin{frame}[fragile]{Memory}
    We can model the memory as an array of bytes
    \begin{tcolorbox}
        \begin{minted}{c}
   uint8_t memory[SIZE];
        \end{minted}
    \end{tcolorbox}
    The ARM is a 32bit architecture and so it may be more
    convenient to model the memory as an array of 32bit words
    \begin{tcolorbox}
        \begin{minted}{c}
   uint32_t wordmemory[SIZE/4];
        \end{minted}
    \end{tcolorbox}

\end{frame}

\begin{frame}[fragile]{Pointer Arithmetic}
Given
\begin{minted}{c}
    uint32_t *word;
\end{minted}
Then\\
\begin{tabular}{ll}
    \mintinline{c}{word} & is an address aligned to 4 bytes \\
    \mintinline{c}{*word} & is an unsigned 32 bit integer at that address \\
    \mintinline{c}{word+1} & is the address of the \emph{next} integer, \alert{4 bytes on} \\
    \mintinline{c}{*(word+1)} & is the \alert{next} 32 bit integer\\
    \mintinline{c}{word[0]} & is the integer \alert{at} the address in \mintinline{c}{word}\\
    \mintinline{c}{word[1]} & is the next integer (at \mintinline{c}{word+1})
\end{tabular}
Pointer arithmetic (and arrays) take into account the \emph{size} of the thing pointer to.

\begin{exampleblock}{See also}
    the \mintinline{c}{sizeof} compile time operator
\end{exampleblock}
\end{frame}

\begin{frame}[fragile]{An initial model}
    We can have an initial model, thinking of the I/O memory as an array of words.

    \begin{tcolorbox}
        \begin{minted}{c}
enum registers {
    Output,Set,Clear,Toggle,Input,Direction
};
uint32_t *PortB = (uint32_t*)0x400FF00;
        \end{minted}
    \end{tcolorbox}

    The ports registers are now simply array access

    \begin{tcolorbox}
        \begin{minted}{c}
    PortB[Direction] |= 1<<22;
        \end{minted}
    \end{tcolorbox}
    Sets bit 22 in Port-B's direction register.
\end{frame}

\begin{frame}[fragile]{Complex declarations}
    The complete declaration for a memory mapped I/O register is something like.
    \begin{tcolorbox}
        \begin{minted}{c}
volatile uint32_t *const IOmap;
        \end{minted}
    \end{tcolorbox}
    Which reads as\ldots
    \mintinline{c}{IOmap} is a \alert{Constant} \alert{Pointer} to an \alert{unsigned 32 bit integer} which is \alert{Volatile}
    \begin{description}
        \item[Constant pointer] the value of the pointer (address) is constant
        \item[Volatile] tells the compiler that the value at the address may change, so always fetch the value from memory.
    \end{description}
\end{frame}


\end{document}

%%%%======================
\begin{frame}{Layer model}

\end{frame}
\begin{frame}[fragile]{An API}
    A device driver:
    \begin{itemize}
        \item opens and initialises a device for use
        \item reads and writes data as appropriate
        \item closes and shuts down the device
    \end{itemize}
    \begin{block}{C stdlib}
        The C library has low level:
        \texttt{open()}, \texttt{read()}, and \texttt{write()}\\
        and higher level \texttt{putchar()}, \texttt{getchar()}, etc
    \end{block}
    An LED will have:
    \begin{itemize}
    \item as write
    \begin{itemize}
        \item turn on
        \item turn off
        \item toggle (change state)
    \end{itemize}
    \item as read (not really meaningful)
\end{itemize}
\end{frame}

\begin{frame}[fragile]{Operation}
    \begin{itemize}
        \item We can access individual bits through the Bit-alias region.
        \item We can have an array of ponters, for each LED, for each function
        \item We can have an array of arrays, one row for each LED.
        \item We'll have to de-reference the pointers to read and write the bits.
    \end{itemize}

    \begin{exampleblock}{}
        \begin{minted}{c}
   uint32_t *LEDregisters[8]; /* 8 pointers for LED control */
   uint32_t  *LEDs[3][8];     /* 3 arrays of 8 pointers     */
        \end{minted}
    \end{exampleblock}

\end{frame}


\begin{frame}[fragile]{Major and minor device numbers}
    Historically Uinx used \emph{major} and \emph{minor} device numbers:
    \begin{block}{From Unix}
    \begin{description}
        \item[Major] number is the class of device, and looks up the functions (row in table)
        \item[Minor] number is the identifier of that particular device
    \end{description}
    \end{block}

    In practice the Major number is used as an index into a table of device drivers, and the minor number is passed as a parameter to the driver.

\begin{exampleblock}{Example code}
\begin{minted}{c}
int read(unsigned int device)
{
    return devtable[major(device)].read(minor(device));
}
\end{minted}
\end{exampleblock}
\end{frame}

\begin{frame}{How to structure Device numbers}
    \begin{itemize}
        \item Possible elements are:
        \begin{description}
            \item[Device] LED, Port, Bit, etc
            \item[ID] 0 \ldots 1
            \item[Connection] Port and bit numbers
        \end{description}
        \item using groups of bytes,
    \end{itemize}

Major and minor numbers are unsigned 16bit numbers,\\
 packed into 32bits.\\[1em]

    \begin{bytefield}[endianness=big]{32}
        \bitheader{0,7,8,15,16,23,24,31}\\
    \bitboxes{16}{ {Major} {Minor}  }
    \end{bytefield}
\end{frame}

\begin{frame}[fragile]{API design and semantics}
    \begin{block}{We have to design the API, it should have the operations}
    \begin{minted}{c}
  open(device, mode);
  read(device);
  write(device, data);
  close(device);
\end{minted}
    \end{block}
We need to decide on data types and semantics
\begin{alertblock}{Semantics}
    Semantics describes the processes a computer follows when executing a program in that specific language.

    In our case, how to interpret the values passed as parameters, and how to interpret the value returned by the function.
\end{alertblock}
\end{frame}

\begin{frame}[fragile]{API design and semantics : BIT}{Example semantics for bits}
\begin{block}{\mintinline{c}{bit = open(bitID, 'r')}}
Opens a bit for reading, the direction bit is set for input.
\end{block}
\begin{block}{\mintinline{c}{bit = open(bitID, 'w')}}
Opens a bit for writing, the direction bit is set for output.
\end{block}
\begin{itemize}
    \item The returned value can be the index into the table of internal addresses, or a negative number to signify an error.
    \item The \alert{\texttt{bitID}} signifies which port and bit number to open.
    \item The open function has to make sure the Port is also open.
\end{itemize}
\end{frame}

\begin{frame}[fragile]{API design and semantics : BIT}{Example semantics for bits}
    \begin{block}{\mintinline{c}{r = write(bit, value)}}
    Writes to a bit, setting it to 1 or 0 as given by \alert{\texttt{value}}.
    \begin{tabular}{rl}
         0  & clear the bit \\
        1  &set the bit\\
         -1 & toggle, change the state of the bit
    \end{tabular}\\
      The \alert{\texttt{bit}} would be the ID returned by \texttt{open()}.  The return value signifies success or failure.
\end{block}
\begin{block}{\mintinline{c}{r = read(bit)}}
    Reads from a bit.  The \alert{\texttt{bit}} would be the ID returned by \texttt{open()}.  The return value gives the value of the bit, 0 or 1.
\end{block}

\end{frame}

\begin{frame}[fragile]{API design and semantics : BIT}{Example semantics for LEDs}
\begin{block}{\mintinline{c}{bit = open(ledID, 'w')}}
Opens an LED.
\end{block}
\begin{itemize}
    \item The \alert{\texttt{ledID}} signifies which LED to open.
    \item The driver opens the appropriate bit for writing
    \item It makes no sense to open an LED for reading!
\end{itemize}
\begin{block}{\mintinline{c}{r = write(led, value)}}
Writes to an LED, setting it as given by \alert{\texttt{value}}.
\begin{tabular}{rl}
     0  & turn off the LED \\
    1  &s turn on the LED \\
     -1 & toggle, change the state of the LED  (flashing)
\end{tabular}\\
\end{block}
\end{frame}

\begin{frame}{Design of device numbers}
\begin{exampleblock}{Ports and Bits}
    In the FRDM-K64F often we have to write to a particular bit on a particular port.
    The \alert{Bit-Alias} region allows access to individual bits.
\end{exampleblock}
\begin{block}{Device registers}
    The FRDM-K64F has registers to: write, set, clear, toggle, and read bits
\end{block}
\end{frame}

\begin{frame}{Semantics of operations}
\begin{tabular}{lll}\toprule
    Open & Port & assign clock signal to port, enabling port \\
         & Other & opens the port device is attached to \\
    \midrule
    Write & Bits & 0 -- clear the bit \\
          &      & 1 -- sets the bit \\
          &      & -1 -- toggles the bit \\

    \bottomrule
\end{tabular}

\end{frame}
\newcommand{\hex}[1]{\mathtt{#1}_{\text{\tiny 16}}}
\begin{frame}[fragile]{Bit-Alias Address Calculations}
\begin{tabular}{ll}
    Port base & $ \hex{000FF000} $ \\
    Port & $ \hex{000FF000} + P\times \hex{40} = \hex{000FF000} + P \ll 6$ \\
    P reg & $\hex{000FF000}+P\ll6+r\ll2$ \\
    Bit offset & $ P_r  \times 32 + b\times4$ \\
             & $ P_r \ll 5 + b\ll2$ \\
             & $(\hex{FF000}+P\ll6+r\ll2)\ll5+b\ll2$ \\
             & $\hex{FF000}\ll5+P\ll11+r\ll7+b\ll2$ \\
\end{tabular}

\begin{description}
    \item[P] 0\ldots4  \texttt{100}
    \item[r] 0\ldots5  \texttt{101}
    \item[b] 0\dots31  \texttt{11111}
\end{description}

\begin{bytefield}[endianness=big]{32}
    \bitheader{31,16,15,13,11,10,9,7,6,2}\\
    \bitbox{16}{1FE} &
    \bitbox{2}{00}& \bitbox{3}{Port} &
    \bitbox{1}{0}&\bitbox{3}{r} &
    \bitbox{5}{bit} &
    \bitbox{2}{00}\\
\end{bytefield}
\end{frame}

\begin{frame}{}

\begin{itemize}
    \item Use Macros in code for readbility
    \item AWK script to calculate in parallel
    \item AWK results used to create Unit tests
\end{itemize}
\end{frame}
