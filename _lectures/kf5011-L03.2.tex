\documentclass[10pt]{beamer}

\usepackage{pgf, tikz, verbatim, amsmath}
\usetikzlibrary{arrows,automata}
\usepackage[latin1]{inputenc}
\title{Analogue Signals}
\author{M J Brockway}

\begin{document}
\maketitle

\begin{frame}
\frametitle{Digital vs Analogue}
Digital (electrical) inputs to a CPU register as logical values - true or false, 1 or 0, {\color{red}on} or {\color{blue}off}.
\begin{itemize}
\item typically arise from switch contacts opening or closing. 
\item eg button press or release
\end{itemize}

A digital output is used to switch something on or off.

Analogue signals vary smoothly over a range of values and may momentarily take any value in between.
\begin{itemize}
\item Eg an audio signal from a microphone; 
\item a voltage from a sensor - temperature, light level, acceleration, force (strain gauge), humidity, pressure, ...
\end{itemize}
\end{frame}

\begin{frame}
\frametitle{Analogue signal}
\begin{figure}[!htb]
\begin{center}
\includegraphics[width=\textwidth]{anlgTrace0.png}
\end{center}
\end{figure}

To input an an analogue signal it has to be \emph{digitised} 
\begin{itemize}
\item \emph{sampled} at a regularly spaced series of times;
\item to produce a series of numbers.
\end{itemize}
\end{frame}

\begin{frame}
\frametitle{Analogue signal}
\begin{figure}[!htb]
\begin{center}
\includegraphics[width=0.8\textwidth]{anlgTrace.png}
\end{center}
\end{figure}

\begin{itemize}
\item An \emph{analogue-to-digital converter} (ADC) is a hardware device that does this.
\item It is configured with a \emph{sampling rate} and an \emph{output resolution}.
\item Output resolution is the number of bits available to store a sample value. With 12 bits you can store values in range 0 to $2^{12} - 1 = 4095$.
\end{itemize}
\end{frame}

\begin{frame}
\frametitle{Analogue/digital conversion}
The opposite process is \emph{digital-to-analogue conversion} (DAC). 
\begin{itemize}
\item From a series of digial values, creates a time-varying analogue signal.
\item An analogue output may drive an audio speaker or a dimmable lamp. (Theatre lights 'synthesize' colours by combining red, green, blue with variable brightnesses.)
\end{itemize}

Digital audio
\begin{itemize}
\item takes analogue input from a microphone (or a mixed audio signal from several), 
\item passes it through an ADC;
\item resulting stream of bits is saved raw (WAV) or in compressed format (FLAC, OGG, MP3).
\item To reproduce the sound, the bit stream is read from the save medium,
\item and passed to a DAC to recreate the audio signal,
\item which (after amplification etc) drived a speaker.
\end{itemize}
\end{frame}

\begin{frame}
\frametitle{ADC techniques}
A \emph{successive approximation converter} first
\begin{itemize}
\item compares the input with a voltage which is half the input range;
\item if input $>$ this level, it compares it with $\frac{3}{4}$ the range;
\item and so on: twelve steps \texttt{=>} 12-bit resolution.
\item During the comparisons, signal is frozen in a \emph{sample and hold circuit}.
\end{itemize}

A \emph{dual-slope integrating converter} 
\begin{itemize}
\item lets the input signal charge a capacitor for a fixed period;
\item then measures the time for the capacitor to fully discharge at a fixed rate;
\item this time is proportional to the `integrated' (averaged over the sample period) input voltage.
\item Slower than successive approximation, but reduces the effects of electrical `noise'.
\end{itemize}

There are other types of ADC which refine these ideas.
\end{frame}

\begin{frame}
\frametitle{ADC techniques}
The \emph{resolution} of an ADC is
\begin{itemize}
\item $n$ \emph{bits}, where the input range is divided into $2^n$ \emph{steps}.
\item Eg a 12-bit ADC will have $2^12 = 4096$ steps;
\item A 0-10 volt input range will then resolve into 2.5 mV steps.
\end{itemize}

\emph{Linearity} of an ADC ...
\begin{itemize}
\item Ideally, with $n$-bits resultion you get $2^n$ steps \emph{of equal size}.
\item In practice, the sizes of the steps vary a little -- non-linearity.
\item \emph{Maximum linearity error} of $n$ percent means the steps vary in size no more than $n \%$ from the ideal step size, $2^{-n}$ of the range.
\end{itemize}

A \emph{sample-and-hold} circuit ...
\begin{itemize}
\item freezes the analogue input voltage at the moment the sample is required,
\item holds it constant while the ADC digitises it.
\end{itemize}
\end{frame}

\begin{frame}
\frametitle{ADC techniques}
\emph{Thoughput} ...
\begin{itemize}
\item The \emph{acquisition time} is the time for the ADC to capture the input voltage during a read; the \emph{conversion time} is time to determine from this the output value (eg by timing a capacitor discharge).
\item \emph{Throughput} = 1/(acquisition time + conversion time).
\item A \emph{pipelined} ADC improves thoughput.
\end{itemize}

An \emph{integrating} ADC, such as the dual-slope ADC
\begin{itemize}
\item times the charge or discharge of a capacity to get an average of the voltage over the sampling cycle.
\item The time to do this is the \emph{integration time}. Convdersion time of a dual-slope converter is a function of this.
\end{itemize}
\end{frame}

\begin{frame}
\frametitle{Digital to analogue conversion - DAC}
This is an electronic circuit which accepts at regular intervals a (digital) \emph{number} at its input and produces a corresponding analogue signal, usually a voltage at its output.
\begin{itemize}
\item Over time, a series of analogue signals are output.
\end{itemize}

These might be voltage or current \emph{control signals} ...
\begin{itemize}
\item Frequency (number of output per second) is low;
\item Outputs determine a motor speed or light intensity or current in a heater or ...
\end{itemize}

They might be to generate a \emph{waveform} ...
\begin{itemize}
\item an audio or video signal for example;
\item frequency can be hundreds to millions of times per second.
\end{itemize}
\end{frame}

\begin{frame}
\frametitle{DAC applications}
\begin{itemize}
\item digital audio, video;
\item high-end instrumentation: waveform genertors, medical imaging;
\item wireless communication systems: mobile phones, satellite terminals, point-to-point and multi-point communication links. 
\item radar systems
\end{itemize}
\end{frame}

\begin{frame}
\frametitle{DAC output range}
 - the maximum and minimum voltage or current that can be generated: 
\begin{itemize}
\item bipolar - eg -5 V to +5 V; or
\item unipolar - eg 0 to 20V.
\item There is often a choice of ranges; choose smallest that will do the job.
\end{itemize}
\end{frame}

\begin{frame}
\frametitle{DAC resolution}
- the number of steps into which the output range has been divided.
\begin{itemize}
\item $n$-bit resolution \texttt{=>} $2^n - 1$ steps ($2^n$ values).
\item For instance DAC with 12-bit resolution divides its output range into $2^12 = 4096$ steps.
\item If the output range is 0-10 V, it is resolved to 2.5 mV steps.
\item Thus, output is not truly analogue: it is stepped!
\end{itemize}
\end{frame}

\begin{frame}
\frametitle{DAC slew rate and settling time}
Slew Rate
\begin{itemize}
\item the maximum rate of change of the output signal:
\item measured by the rise in voltage divided by time
\item Eg volts per microsecond.
\end{itemize}
\end{frame}

\begin{frame}
\frametitle{DAC settling time}
\begin{itemize}
\item When the DAC changes from its minimum output level to its maximum, the output signal swings through its \emph{full scale}.
\item The settling time indicates how long it will take the output to settle to
its final voltage
\item time to settle to a percentage of the full-scale voltage or current range, following a full-scale change.
\item actual output wobbles about for a few microseconds before setting ...
\end{itemize}

\begin{figure}[!htb]
\begin{center}
\includegraphics[width=0.8\textwidth]{settling.png}
\end{center}
\end{figure}
\end{frame}

\begin{frame}
\frametitle{The Nyquist criterion}
How do you decide the sampling rate of an ADC?
\begin{itemize}
\item You want to know that you will get an accurate copy of the signal when you feed the data to a DAC!
\item The \emph{Nyquist criterion} says: sample at \emph{twice the bandwidth} of the original signal: twice the highest frequency present in the original signal.
\item This guarantees you enough data to rebuild a fair copy of the signal with a DAC, provided ...
\item you feed the rebuilt signal through a \emph{filter} - an electronic circuit which reject frequencies outside the band you are interested in.
\item This is based on \emph{Fourier theory}, a mathematical theory that shows how any waveform with a maximum frquency $f$ can be built of `sine waves' of frequencies up to $f$. An ADC datum for each half-cycle at the maximum frequency will do the trick, according to Nyquist.
\end{itemize}
\end{frame}

\begin{frame}
\frametitle{The Nyquist criterion - examples}
Use a sampling rate of $2.2 \times f$ max to allow for practical filters.

Landline telephony supports audio for speech conversation in the range 300 to 3400 Hz. 
\begin{itemize}
\item sampled at 8 kHz
\end{itemize}

`CD quality' audio is based on the idea that we hear sounds up to 20 kHz.
\begin{itemize}
\item CD quality sampling rate is 44.1 kHz
\item CD is recorded in stereo and each channel uses a 16-bit ADC....
\item Combined ADC output is 1411200 bits/sec: 10.582 Mb/min.
\item a nominally 700 Mb compact disk will support around 66 minutes of playing time.
\end{itemize}
\end{frame}

\begin{frame}
\frametitle{The Nyquist criterion - aliasing}
If sampling is at a \emph{lower} frequency that demanded by the Nyquist criterion, i.e. at less than twice the maximum frequency in the input waveform, then
\begin{itemize}
\item the sum and difference components associated with each \emph{harmonic} of the input waveform overlap with those of adjacent harmonics and
\item the sampled waveform can no longer be separated out by filtering.
\end{itemize}

This is a bit technical (mathematical Fourier theory again) but the effect is that the waveform reconstituted by the DAC (in your CD player for instance) will not make a faithful copy of the original waveform.

\medskip
A slightly mathematical discussion of the Nyquist criterion is to be found at \texttt{\color{blue}https://en.wikipedia.org/wiki/Nyquist\_ISI\_criterion} and in the same spirit, the article on aliasing is also worth a read: \texttt{\color{blue}https://en.wikipedia.org/wiki/Aliasing}.
\end{frame}


\end{document}
